\documentclass[11pt]{jarticle}
\usepackage[a4paper, margin=20mm]{geometry}
\usepackage{jtygm}

\title{SamurAI Coding 2017-18 予選ルール}
\author{情報処理学会プログラミングコンテスト委員会}
\date{2018/01/09}

\begin{document}
\maketitle

\begin{abstract}
  SamurAI Coding 2017--2018 コンテストの予選の実施と決勝進出者の決定方法を定める.
  このルール案は暫定版であり, 詳細については今後の改訂の可能性がある.
\end{abstract}

\section{予選の構成}

参加チーム数が総当り戦を実施可能な範囲ならば, すべての参加チームが他の
すべての参加チームと1ゲーム (スタート位置を交換した2レース) ずつを競う
総当り戦によって行う.  参加チーム数が多く単純な総当り戦が実施困難であ
る場合は, 2次からなる予選を行う.

2次からなる予選を行う場合, 1次予選では参加チームをランダムにいくつかの
グループに振り分け, 各グループで総当り戦を実施する.  1次予選の各グルー
プ上位者を30チーム程度選抜し, 総当り戦での最終予選を行う.

実施に要する手間の制約から, 1次予選の各グループのチーム数は最大30チー
ム程度となるよう, 1次予選のグループ数を調整する.
1次予選の各グループのチーム数はできる限り均一にする.
また, 各グループから最終予選に進出するチームは同一とする.

下表に参加チーム数に応じた予選の構成例を示す.

\begin{table}[h]
  \begin{center}
    \begin{tabular}{r|rrrr|rr}
      \multicolumn{1}{c|}{参加}&\multicolumn{4}{c|}{1次予選}&\multicolumn{2}{c}{最終予選}\\
      チーム数&グループ数&チーム数&進出数&試合数&チーム数&試合数\\
      \hline
      100&4&26&8&1300&32&496\\
      150&10&16&3&1200&30&435\\
      200&15&14&2&1365&30&435\\
      300&30&10&1&1350&30&435\\
    \end{tabular}
  \end{center}
\end{table}

\section{総当り戦の方法}
$n$チームによる総当り戦は, 毎回対戦相手を変えた$n-1$ステージにより行う.
チーム数が奇数である場合は, 主催者が用意したプレイヤを追加して偶数にし,
全チームが全ステージで異なる相手と各1ゲームを行うようにする.

各ステージの全ゲームは同一のコースを用い, ステージごとには異なるコース
を用いる.

総当り戦の順位は, 以下の項目をこの順序の優先順位で適用し, 最大の者を上位とする.
\begin{enumerate}
\item
  合計勝点.
  各ステージにおいて, ゲームの勝者に2点, 敗者に0点, 引分の場合には両者に1点を与え,
  全ステージの勝点を合計したもの.
\item
  合計タイム.
  全ステージの全ゲームの両レースについて, ゴールタイムを合計したもの.
  ゴールタイムはゲームルールに定義するものである.
\end{enumerate}
上記の項目を適用して同順位となる場合には, 抽選により順位を決する.

\section{レースコース}
予選に用いるレースコースは以下を満たす.
\begin{itemize}
\item コースの長さは50以上100以下である.
\item コースの幅は2以上20以下である.
\item 視界は5以上である.
\item 考慮時間は200msに制限ステップ数を乗じ1000msを加えたものである.
\end{itemize}

\section{決勝進出チーム}
予選上位 (予選を2次に分けて行う場合は, 最終予選上位) の12チーム以上を
決勝進出チームとして選抜する.

これらのチームの他に, 地域などの多様性と予選の戦績等を考慮して, 最大4
チームを選抜し, 決勝進出は計16チームとする.

\begin{flushright}
以上
\end{flushright}

\end{document}
