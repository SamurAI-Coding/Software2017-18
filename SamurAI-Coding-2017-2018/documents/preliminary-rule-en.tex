\documentclass[11pt]{article}
\usepackage[a4paper, margin=20mm]{geometry}

\title{SamurAI Coding 2017-18 Qualifying Round Rules}
\author{IPSJ Programming Context Committee}
\date{2018/01/09}

\begin{document}
\maketitle

\begin{abstract}
  This document describes the system of the qualifying round and the
  advancement criteria for SamurAI Coding 2017--2018 contest.

  The rules described here are provisional and subject to change.
\end{abstract}

\section{Preliminary Round System}

The qualifying round is conducted as a round-robin tournament, in
which all the participating teams play one game (two races exchanging
the start positions) against all the other teams, as long as the
number of participating teams allows it.  If the number of teams is
too many for a single round-robin tournament, the qualification round
will be organized with two rounds.

In a two-round qualifier, participating teams are divided into several
groups for the first round.  A round-robin tournament is conducted in
each of the groups.  Around 30 teams with higher ranks in the
first-round groups are advanced to the final qualifier, which is
conducted as a round-robin tournament again.

Due to the limited time and resource, the numbers of teams in each of
the first-round qualifier are restricted to at most around 30.  The
number of teams for the first-round groups are averaged as far as
possible.  The same number of teams are advanced to the final
qualifier from each group.

Below is an example of the organization of the qualifying round with
different number of participating teams.

\begin{table}[h]
  \begin{center}
    \begin{tabular}{r|rrrr|rr}
      \multicolumn{1}{c|}{Total number}&\multicolumn{4}{c|}{First Round}&\multicolumn{2}{c}{Final Qualifier}\\
      of teams&groups&teams&advanced&games&teams&games\\
      \hline
      100&4&26&8&1300&32&496\\
      150&10&16&3&1200&30&435\\
      200&15&14&2&1365&30&435\\
      300&30&10&1&1350&30&435\\
    \end{tabular}
  \end{center}
\end{table}

\section{Round-Robin Tournaments}
A round-robin tournament with $n$ teams consists of $n-1$ stages, each
with different opponents.  When the number of teams in the tournament
is odd, a player provided by the organizer is added to make it even,
making each of the team play against all the other teams.

All the games in each stage use the same race course, and different
courses are used in different stages.

The ranks of a round-robin tournaments are decided according to the
following criteria, in this order.
\begin{enumerate}
\item
  Total points.  In each stage, game winners are given two points and
  losers are given no points.  When the game is drawn, both will be
  given one point.
\item
  Total time.  The total of the goal time of all the races of all the
  stages.  The goal time is defined in the game rules.
\end{enumerate}
When two or more teams are ranked the same with the above criteria,
ranks are decided by drawing lots.

\section{Race Course}
Race courses used in the qualifying round satisfy the following.
\begin{itemize}
\item The course length is between 50 and 100, inclusive.
\item The course width is between 2 and 20, inclusive.
\item The vision limit is greater than or equal to 5.
\item The initial remaining time is 200ms times the step limit plus 1000ms.
\end{itemize}

\section{Advancement to the World Finals}
Twelve or more higher-ranked teams in the qualification round (in the
final qualifier, when the qualification is organized with two rounds).

At most four teams are selected in addition, considering regional
diversity, results in the qualifier round, etc., and sixteen teams in
total are advanced to the world finals.

\begin{flushright}
以上
\end{flushright}

\end{document}
