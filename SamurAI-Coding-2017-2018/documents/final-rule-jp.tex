\documentclass[11pt]{jarticle}
\usepackage[a4paper, margin=15mm]{geometry}
\usepackage{graphicx}
\usepackage{wrapfig}
\usepackage{jtygm}

\title{SamurAI Coding 2017-18 決勝ルール}
\author{情報処理学会プログラミングコンテスト委員会}
\date{2018/02/16}

\begin{document}
\maketitle

\begin{abstract}
  SamurAI Coding 2017--18 コンテストの決勝の実施方法を定める.
\end{abstract}

\begin{wrapfigure}{r}{0.52\columnwidth}
  \begin{flushright}
  \vspace{-2cm}
  \includegraphics[width=0.5\columnwidth, natwidth=4.88in, natheight=3.6467in]{tournament-table-jp.png}
  \vspace{-1.5cm}
  \end{flushright}
\end{wrapfigure}

\section{決勝の構成}

決勝は予選を通過した16チームによる勝ち抜きトーナメントにより行う.
これに加えて, 準決勝で敗退した2チームが対戦する三位決定戦も実施する.

トーナメントの各ゲームでは1ゲーム (同じコースでスタート位置を交換した2レース) を競う.
ゲームが引分だった場合は, 異なるコースを用いて再戦を行う.
再戦がまた引分だった場合は, シード順がより上位であったチームが勝ち進むものとする.

勝ち抜きトーナメントは右に示す表に従って行う.
番号は参加チームのシード順を表す.

\section{レースコース}
決勝に用いるレースコースは以下を満たす.
\begin{itemize}
\item コースの長さは100以上200以下である.
\item コースの幅は2以上20以下である.
\item 視界は5以上である.
\item 考慮時間は200msに制限ステップ数を乗じ1000msを加えたものである.
\end{itemize}

決勝1回戦の8ゲーム, 準々決勝4ゲーム, 準決勝2ゲーム, 決勝と三位決定戦の
2ゲームについては, それぞれ同一のコースを用いる.

\section{シード順}
シード順は予選順位に基づき以下の通りに定める.
\begin{itemize}
\item
まず予選上位チームを予選順位通りにシードする.
\item
次いで, 特別選抜枠のチームを予選順位順にシードする.
\end{itemize}
複数チームが予選において同順位である場合, そのシード順は抽選により定める.

\begin{flushright}
以上
\end{flushright}

\end{document}
